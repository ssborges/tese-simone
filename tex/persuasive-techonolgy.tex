%-------------------------------------------------------
% The Persuasive Technology Theory
%-------------------------------------------------------
\section{The Persuasive Technology Theory}
\lipsum[1]

%-------------------------------------------------------
% Persuasion Profiling
%-------------------------------------------------------
\section{Persuasion Profiling}

Persuasive technology is an approach that intends to change user’s attitudes and/or behavior using persuasion [1]. Persuasive technology feasibility has increased thanks to recent computing advances and the surge of interest in the area. Evidence that persuasive technologies can persuade users can be found in [2]–[6]. In intelligent learning environments, persuasive technologies have been used to increase students’ engagement and to reduce the feeling of obligation towards executing pedagogical tasks [7]. However, as pointed out by [8], a reliable use of persuasion strategies involves delivering the right message in a specific way at the precise moment. Yet, trying to figure out what is “the right message” for a student and how to deliver it at the “right time” are still difficult tasks. 
As shown in [9] and in [10], the design and implementation of persuasive (learning) systems are complex tasks since it is hard to estimate the effectiveness of those strategies regarding each learner. As stated by [11], many social psychologists have pointed out that one sound way to improve the effectiveness of persuasive attempts relies on the design of personalized persuasive strategies, tailored to fit user’s personality traits. Thus, we can design the “right message”, and deliver it at the “right time” to the “right user”. Nevertheless, most research still focus on the target benefits that can be achieved using persuasive systems, and less research have been found on how to design effective approaches tailored to different users in order reach such benefits [2], [3], [12]. Significant contributions have been proposed in the last decade [3], [5], [7], [13]. Among them, Kaptein et al. introduced the use of persuasive profiles in Human-Computer Interaction (HCI) contexts in 2009 [8]. Later, the benefits of using persuasive profiles have been examined in several studies [6], [11], [14], [15]. 
The design of persuasive profiles relies primarily on measuring user’s sensibility to persuasive strategies [6], thus addressing the challenge of enabling the personalization of persuasive attempts. Secondly, delivering personalized content based on user’s profile. Since the design of effective persuasive profiles demands the capacity to measure user’s susceptibility to persuasive strategies (i.e. influence principles [16]), Kaptein et al. have perceived the need to develop psychometric instruments to measure user’s responsiveness to persuasive strategies in a systematic way [4], [5], [11]. They developed and validated a 26-Item questionnaire called Sensibility to Persuasion Scale (STPS) [6]. 
Besides STPS, we found two initiatives in developing measurement scales in the persuasive technology literature. Firstly, in Modic et al. [5], a generalized modular psychometric tool to measure individual susceptibility to persuasion (StP-II) was evaluated. One differential mentioned by the authors is that their scale covers a wide scope of persuasive strategies. They ran an exploratory factor analysis and reliability test on StP-II (N=279 subjects) with 10 subscales, and covering 54 items. According to the authors, the tests confirmed the internal reliability of the scale. Secondly, in Busch et al. [4], instead of covering a wide range, the authors developed and validated an inventory for measuring sensibility to persuasion of a subset of persuasive strategies. They performed some analysis and reliability tests using data from 167 subjects. The original scale has five subscales, covering 40 items. According to the authors, the tests confirmed the reliability of only three subscales, while two others did not show enough internal consistency. Therefore, the later subscales were excluded from the final version.
Finally, the third scale found was STPS [11]. STPS is based on the six social influence strategies compiled by Cialdini [16]. Although still there is no consensus about the number or influence principles [6], in his book, Cialdini has demonstrated numerous positive usages of such compilation. STPS is an explicit approach to profile users. Explicit profiling is a meta-judgmental measurement that invite users to think and to reflect upon their own personal traits [6]. Using the results of these measurements, designers of persuasive technology can extract useful information to identify effective persuasive strategies that can be helpful during the design and development of, for example, persuasive learning systems. A well-known limitation of such approach consists in the user’s susceptible to socially desirable answers effect [4]. However, when users’ answers are interpreted in association to demographics data and operative measures, the information gathered still can help designing better solutions then the “one size fits all” solution, commonly found in the majority of the persuasive technology attempts [11].

\todo[inline]{não esquecer de escrever um parágrafo sobre ética remetendo ao persuasive technology do Fogg}

%-------------------------------------------------------
% Influence Principle
%-------------------------------------------------------
\subsection{Influence Principles}


%-------------------------------------------------------
% Sensibility to Persuasion Scale
%-------------------------------------------------------
\subsection{Sensibility to Persuasion Scale}

%-------------------------------------------------------
% Concluding Remarks
%-------------------------------------------------------
\section{Concluding Remarks}

% http://images.pearsonassessments.com/images/tmrs/Motivation_Review_final.pdf
% Do I want to do this task and why?
% A separate body of research within the study of motivation has focused on answering the
% question, Do I want to do this task and why? Under this category, Broussard and Garrison (2004)
% include expectancy-value theories, intrinsic motivation theories, and self-determination theory. 
