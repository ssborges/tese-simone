%---------------------------------------------
% The Systematic Mapping Process
%---------------------------------------------
\section{The Systematic Mapping Process}

As a research area advances, the amount of studies published in that area tends to increase. Thus, obtaining an overview of an established research area can be a complex undertaking. Evidence-based Software Engineering (Dybå et al. 2005) proposes a set of guidelines to support the conduction this sort of investigation. More specifically, these guidelines outline how to identify, evaluate, interpret, and analyze primary studies (Dybå and Dingsøyr 2008). The main assumption is that those guidelines, which are expressed and embodied in systematic mappings and systematic reviews, lead to more consistent results that can be more easily replicated.
Therefore, to obtain a bird's eye view of the approaches that have been used for group formation within CL environments, we carried out a systematic mapping study. Systematic mappings are also known as scoping reviews (Pretorius and Budgen 2008), and they involve methodologically searching the literature to ascertain the nature and broadness (i.e., type and amount of primary studies) of the existing, published research on a particular topic (Magnisalis et al. 2011). In research, primary studies are examples of original research (Kitchenham 2004)

The systematic mapping herein described is based on the guidelines proposed by (Petersen et al. 2008) According to (Petersen et al. 2008), the key steps of a systematic mapping are the following: (i) definition of research questions (RQs), (ii) searching for relevant studies, (iii) screening of studies, (iv) keywording of abstracts, and (v) data extraction and mapping. Figure 1 illustrates these steps as well as the order in which they are performed. As shown in Figure 1, the conduction of each step results in an intermediate result. The main contribution of a systematic mapping is made up of these intermediate results. The following subsections briefly describe how we carried out the steps in Figure 1 and the intermediate outcomes of these steps.

\textbf{Figure 1. Systematic mapping main steps (Petersen et al., 2008).}

%---------------------------------------------
% Definition of the Research Questions
%---------------------------------------------
\subsection{Definition of the Research Questions}
 
As mentioned, the main goal of this systematic mapping is to gather data on and look at the state-of-the-art of research on group formation in the context of CSCL environments. It is worth mentioning that the RQs in a systematic mapping are framed in such a way as to drive the selection of primary studies and the analysis that takes place in the subsequent steps. The RQs that drove this systematic mapping are outlined in Table 1.

\textbf{Table 1. Research questions on the study.}

During this step we also defined the scope of our systematic mapping, outlining its population, intervention and expected results: 
Population: primary studies examining varying aspects of the formation of groups in the context of CSCL.
Intervention: primary studies that either apply, discuss, or propose strategies, approaches methods, or techniques for group formation or report on experiences with group formation.
Expected Results: an overview of the state-of-the-art of group formation. It is expected that, apart from providing a summary of what has been investigated in this research area, the results contribute to the identification of new avenues for research on group formation applied to CSCL by highlighting areas require more research.

%---------------------------------------------
% Searching for Relevant Studies
%---------------------------------------------
\subsection{Searching for Relevant Studies}

We created a search string by combining some terms, their synonyms, and acronyms. The initial set of terms was defined after an interview with a specialist and based on the preliminary scanning of 10 primary studies manually selected. The final set of terms is shown in Table 2. The search encompassed the electronic databases that are deemed as the most prominent scientific sources and, therefore, prone to include relevant primary studies (Dybå et al. 2007). We searched the following electronic databases: ACM Digital Library, IEEE Xplore, Elsevier (via ScienceDirect), Springer (via ScienceDirect), Wiley Online Library, Web of Science, EngineeringVillage and SCOPUS. It is worth mentioning that we did not place any restrictions on date of publication when searching for primary studies. 

\textbf{Table 2. Terms used in electronic libraries for the research of systematic mapping}

Table 3 presents the total amount of papers returned, the number of candidate studies after skimming through titles and abstracts, and the final set of primary studies. The next subsection elaborates on how we arrived at the final set of primary studies.

\textbf{Table 3. Total of returned papers, selected candidates, and the final set.}

%---------------------------------------------
% Screening
%---------------------------------------------
\subsection{Screening}

The purpose of this step is to assess the suitability of primary studies with respect to the RQs. Therefore, in this step; all papers are evaluated against predefined inclusion and exclusion criteria that reflect the RQs. After applying these criteria (Table 4) to titles and abstracts, the initial set containing 3571 papers was reduced to 423 candidate studies. After going over the candidate set, we ended up with 106 primary studies (Figure 2).

Figure 2. Overview of the systematic searching process.

Table 4. Inclusion and exclusion criteria for screening of returned items.

Figure 3 shows the frequency of research on group formation in the context of CSCL over time. As can be seen in Figure 3, the formation of groups has been investigated for approximately 20 years. Moreover, the amount of papers on this topic has dramatically increased since 2006. In particular, 2009 and 2011 stand out as the years in which most efforts in this area have been published so far: 16 studies in 2009 and 17 in 2011. 

Figure 3. Frequency of selected primary studies distributed by year of publication.

%---------------------------------------------
% Keywording
%---------------------------------------------
\subsection{Keywording}

This step is centered on classifying and categorizing primary studies. Initially, the abstracts of the primary studies were scanned for keywords that reflect the contributions presented in these studies. Thereafter, the resulting keywords were merged (i.e., terms that were considered synonyms were collapsed into the most commonly occurring term between the two) and analyzed in hopes of gauging the nature of the contributions in this area. 
We use the keywording to classify the studies according to our framework and answer the RQs that we set out to investigate. Note that primary studies can fall into more than one category in our framework; thereby some of the proposed categories overlap.

%---------------------------------------------
% Systematic Mapping Results
%---------------------------------------------
\section{Systematic Mapping Results}

In this section, we analyze the results of our systematic mapping. We draw information from the primary studies in order to answer the RQs of our mapping study. 

%---------------------------------------------
% \subsection{Research Question 1: What are the most investigated characteristics of group formation in the domain of CSCL?}
%---------------------------------------------
\subsection{Research Question 1: What are the most investigated characteristics of group formation in the domain of CSCL?}

The categories, as mentioned earlier, were created with the purpose of shedding some light on what are the most investigated characteristics of group formation in the domain of CSCL. By characteristics we mean that we applied a keywording strategy to prepare our rating system and characteristics for the selected primary studies. By applying such a strategy, initially summaries were read to find keywords and concepts that reflect the study's contribution. Next, these keywords and concepts were combined to produce a general understanding of the nature and contributions of the research. The classification scheme gradually evolved towards its final version as some characteristics were eliminated, added, merged or split (Durelli et al. 2010).
Also, by grouping the primary studies into categories allowed us better to grasp the interrelationships among categories and how efforts in this area overlap. However, it is worth mentioning that the resulting categories do not encompass all characteristics, research themes and contributions in the area. Rather, the categories were devised in hopes of better answering the RQ1.

%---------------------------------------------
% \subsubsection{Group Formation Planning}
%---------------------------------------------
\subsubsection{Group Formation Planning}

As mentioned, this category groups the primary studies in terms of the sort of planning that was used during group formation. Here we classify as unsystematic group formation approaches within which the formation of groups lacks a set of governing rules and takes places in a seemingly random fashion. In contrast to unsystematic group formation, systematic group formation is based on criteria that govern the size and organization of groups. Figure 5 shows the proposed classification, highlighting the basic difference between systematic and unsystematic group formation. In 12 of the selected studies, students are grouped in a random fashion. Nevertheless, in only 6 of these studies randomness is the only factor determining group formation. In the other studies, it was also possible to group students based on criteria aimed at achieving the target educational objectives. As an example, in the study conducted by (Mujkanovic and Lowe 2012), initially the groups were unsystematic formed (i.e. by chance), given that not enough information to warrant the application of more complex criteria about the students was collected yet. Later, after the conduction of more activities, the students were evaluated, and the results of the evaluations were then used as a group formation criterion. According to (Mujkanovic and Lowe 2012), this approach could be an alternative to using placement or pre-evaluation questionnaires. One drawback of using questionnaires before collaborative learning activities is that carrying out this sort of evaluation takes a significant amount of time, and it is often seen as tedious for the students taking part in it. 

Figure 5. Classification scheme of the primary studies in accordance with in terms of the sort of planning that was used during group formation.

Most primary studies, however, report on group formation approaches that are based on some criteria: in 94 studies, criteria are used to support the definition of rules that establish which group each student should be added to. Figure 6 shows the frequency of the criteria for the formation of groups found in primary studies analyzed. The number of criteria present in each study varies according to the objectives of the collaborative activities to be performed or can be based on the kind of computational support employed (e.g. algorithms). Abnar et al. (2012) remark that, in general, when groups are manually created, only one criterion is used. They also point out that the more criteria are used, the more complicated it is to handle the information related to groups. Nevertheless,  Medina and Gómez-Pérez (2013), and Hwang et al. (2008) state that it is unlikely coming up with one criterion that yields good groups. It follows that it is desirable to have multiple criteria, given; the more information is taken into account during group formation, the greater the control over the quality and granularity of the resulting groups. Recent research indicates that sound pedagogical theories should underpin group formation approaches. For instance, the studies carried out in Isotani et al. (2009) and Isotani et al. (2013) provide further evidence in this regard. Moreover, the approaches proposed in Isotani et al. (2009) and Isotani et al. (2013) benefit from Semantic Web to formalize the knowledge required for creating groups. 
	The most used criterion for group formation was knowledge level. Knowledge level-based group formation is investigated in 59 primary studies. Skill-based group formation is the second most investigated criterion, investigated in 33 primary studies. Learner roles are the third most investigated criteria, 31 primary studies. As shown in Figure 6, the other criteria that have been explored are group objectives, individual objectives, students genre, cultural aspects (e.g., nationality and language), interaction issues, characteristics extracted from social relationships, psychological characteristics, learning disabilities, schedule compatibility, age, and academic formation. It is worth mentioning that these are only some of the identified criteria, studies might have used other additional criteria. However, these other criteria were not mapped because the studies failed to describe sufficiently them. Usually, these additional criteria are simply mentioned in the primary studies as “other criteria” or “among other things”.

Figure 6. Frequency of the criteria for the formation of groups found in primary studies analyzed.

%---------------------------------------------
% \subsubsection{Initiative in Systematic Group Formation}
%---------------------------------------------
\subsubsection{Initiative in Systematic Group Formation}

When group formation follows a methodical (viz., systematic) approach to group formation, the initiative for group formation can be handled in two different ways: spontaneous group formation (i.e., when the students are the ones in charge, yielding self-selected groups) and controlled group formation (i.e., when the instructor or underlying approach controls group formation, resulting in instructor-select groups). This distinction is highlighted in Figure 7. We classify as spontaneous the approaches in which students are free to form their own groups. Spontaneous group formation approaches take into account the preferences of the students (e.g., affinity between students, idiosyncrasies, and friendship). Therefore, we classified as controlled group formation (guided) the approaches that take some other elements into account during the formation of groups, apart from students’ preferences. We argue that the term controlled better describe such approaches, given that there is an external element, which can be impervious to the students’ preferences, effectuating the rules and controlling how students should be grouped.
 
Figure 7. Classification of primary studies in accordance with the initiative of group formation.

In 21 of the selected studies, the preference of the studies was taken into account during group formation. Nevertheless, in only four studies this was the only factor taken into consideration. In the other 17 studies, apart from the students’ preferences, other criteria could also be used. In these primary studies, the students are polled about their preferences, and the gathered information is then used to devise group formation criteria.
The Venn diagram shown in Figure 8 illustrates how the primary studies were categorized in terms of their planning strategy. As mentioned, some primary studies can fall in more than one category since their approaches are flexible enough to allow for more than one planning strategy. By analyzing Figure 8, it can be seen that most group formation approaches (70 primary studies) are exclusively criteria-based approaches.

Figure 8. Distribution of the primary studies according to the criteria used during the clustering strategy.

%---------------------------------------------
% \subsubsection{Population Diversity}
%---------------------------------------------
\subsubsection{Population Diversity}

In the context of this study, diversity has to do with the makeup of groups, which can be heterogeneous or homogeneous as shown in Figure 9. Homogeneous approaches emphasize criteria that result in groups of like-minded students, i.e., students that present similar characteristics. Heterogeneous approaches, conversely, propose the creation of groups that are comprised of members with different backgrounds and characteristics. 

Figure 9. Classification scheme for the audience diversity when forming groups.

The most used approach is the formation of heterogeneous groups (84 primary studies). Evidence suggests that heterogeneous groups foster the interaction among group members due to the members have skills that are complementary to the group as a whole (Barkley et al. 2005; Johnson and Johnson 1999). Also, in heterogeneous groups it is normal to have members with characteristics that supplement other individuals in the group. 
In 36 primary studies relied on the formation of homogeneous groups. Note that in two of these 36 studies the formation of homogeneous groups is the only option, the other 34 studies also propose criteria that yield heterogeneous groups (Figure 10). Also, nine papers were not classified in terms of diversity, given that in five of them the formation of groups can be carried out in a random fashion and four in a spontaneous way. Apart from these papers, we also analyzed 17 primary studies that either use or investigate group formation. However, it was not possible to identify the diversity of the groups in these papers.

Figure 10. Primary studies according to diversity of the population in the groups.

In the study conducted by Chiu and Hsiao (2010), it is argued that randomly assembled groups can result in an acceptable level of heterogeneity with respect to certain characteristics as, for instance, the amount of students of the same gender or with some particular personality traits (e.g., active students, passive students, leaders, and followers). Moreover, in the study conducted by Ardaiz-Villanueva et al. (2011), problem-based learning (PBL) principles and several support tools were used to support the conduction of collaborative learning activities. Two criteria drive group formation: the first is based on a “creativity score” attributed to each student and the second boils down to computing an “affinity score” that gauges how each student gets along with others. According to the authors, the results seem to suggest that homogeneous group (in terms of creativity) and whose members interact with each other well lead to a better learning experience. Besides, the authors report that in this kind of groups they did not detect freeloaders and there was an increase in students’ engagement during collaborative learning activities.

%---------------------------------------------
% \subsubsection{Distribution Approaches}
%---------------------------------------------
\subsubsection{Distribution Approaches}

We categorized the studies according to the way the students were distributed in groups. As mentioned, groups can be homogeneous or heterogeneous groups according to the criterion (or criteria) used. In terms of distribution, we classified the studies whose approaches result in either homogeneous or heterogeneous groups (despite how many criteria were used) as simple distribution. By the other hand, we also found studies in which a subset of criteria was used to cluster the students in homogeneous groups, while at the same time; another subset of criteria was used to cluster them in heterogeneous groups. This second type of approach was classified as hybrid distribution, and the classification scheme is shown in Figure 11.

Figure 11. Simple distribution results in homogeneous OR heterogeneous groups, while in hybrid distribution, groups are concomitantly homogeneous AND heterogeneous according to each subset of criteria.

%---------------------------------------------
% \subsubsection{Computational support}
%---------------------------------------------
\subsubsection{Computational support}

Although the domain of computer-supported collaborative learning contains the words "computer-supported", not always the group formation occurs with technological support. Based on these observations, we devise the category shown in Figure 12, to classify the primary studies according the computational support. 

Figure 12. Classification of the primary studies according to the computational support.

We observed that a few primary studies (13) did not provide enough information clarifying neither the use nor the extension of the computational support, getting this restricted to the other stages of collaborative learning (i.e. collaborative tasks to be performed). In the remaining 83 studies analyzed, various technological solutions have been used. Although the extent and the way these technologies were employed are relevant, they are beyond the scope of this paper and will not be detailed. Table 5 presents an overview of the related major computational resources identified in the studies analyzed, and Table 6 presents the main types of algorithms described in the primary studies. 

Table 6. Major algorithms used to support group formation in the primary studies investigated.

%---------------------------------------------
% \subsubsection{Rationale Behind the Strategy for Group Formation}
%---------------------------------------------
\subsubsection{Rationale Behind the Strategy for Group Formation}

The primary studies were also classified in terms of how the authors back up their approach for group formation. We found three types of studies: (i) studies that do not present any arguments that back up their approaches, (ii) studies whose group formation approach is borne out by empirical evidence, (iii) and studies that draw on assumptions from pedagogical theories (Figure 13). 
 
Figure 13. Classification of the primary studies regard the kind of rationale found in the primary studies explaining the group formation approach.

Only 37 of the 106 analyzed studies are based on sound pedagogical or instructional theories, which back up the selection criteria employed during group formation. We also found that 46 studies do not report whether any pedagogical theory was used to support the group formation criteria. These 46 studies mention, however, that the chosen criteria are based on empirical evidence. Apart from these primary studies, 23 studies do not elaborate on what theories ground their criteria. Table 7 shows an overview of the rationales behind the examined group formation approaches.

Table 7. Regard the kind of rationale found in the primary studies explaining the group formation approach.


%---------------------------------------------
% \subsubsection{The answer for RQ1}
%---------------------------------------------
\subsubsection{The answer for RQ1}

RQ1: What are the most investigated characteristics of group formation in the domain of CSCL?

A\textsubscript{RQ1}:: The majority of the primary studies (101 out of 106) report on approaches for planned group formation, i.e., the formation of groups is based on pre-established criteria. The most widely used criterion is the level of knowledge of the students, which is used in 46 of the selected studies. Controlled group formation is used in 90 primary studies. In terms of distribution, most primary studies rely on heterogeneous groups: 84 studies propose the formation of heterogeneous groups. Nevertheless, only in 50 out of these 84 studies simple heterogeneous groups are the only possible outcome. According to our results, 82 studies capitalize on some sort of technology to automate and support group formation. Only 37 studies exploit educational or pedagogical theories to justify their rationale for the group formation approach. 

%---------------------------------------------
%\subsection{Research Question 2: In what educational level (or learning activities) group formation has been most investigated and applied to?}
%---------------------------------------------
\subsection{Research Question 2: In what educational level (or learning activities) group formation has been most investigated and applied to?}

In order to answer RQ2, we classified the selected studies in terms of their target population. In the context of this paper, target population refers to the stage (or level) at which educational activities take place or the type of educational activities. As shown in Figure 14, the group formation approaches proposed in most studies (56) emphasize tertiary education (viz., undergrad students). Few studies focus on applying group formation approaches to other educational levels (or stages). For instance, our results would seem to suggest that there has been scant research on the impact of group formation when applied to early childhood education (1), graduate students (1), e-Health students (1), language-learning activities (1), primary education students (3), and secondary education students (6). 36 primary studies fail to specify the target population of their proposed approach. Also, these studies present theoretical models that represent solutions that have not been implemented or fully tested yet.
 
Figure 14. Distribution of primary studies according to the level/type of educational activity.

%---------------------------------------------
% \subsubsection{The answer for RQ2}
%---------------------------------------------
\subsubsection{The answer for RQ2}

RQ2: In what educational level (or learning activities) group formation has been most investigated and applied to?

A\textsubscript{RQ2}:: Most research on group formation applied to CSCL has been geared towards undergrad students: approximately 52\% (56 primary studies) of the studies tackle group formation at this educational stage (level). 

%---------------------------------------------
%\subsection{Research Question 3: In what educational level (or learning activities) group formation has been most investigated and applied to?}
%---------------------------------------------
\subsection{Research Question 3: What type of research is the most common in the field?}

Our answer is based on the classification proposed in (Wieringa et al. 2005), which comprises the following research types. Validation Research, studies that fall into this category describe a novel technique, approach, or strategy and have not been implemented in practice, but its effectiveness has been evaluated to some degree through laboratory studies. Evaluation Research, this category contains studies that empirically evaluate a technique, approach, or strategy in practice or real settings. Position Papers, these studies report the authors' point of view. Research in this category does not contain evidence that backs up the authors' opinion. Philosophical Papers, studies that present a new of perceiving things or new guidance for a domain, often using taxonomies and conceptual frameworks. Solution Proposals, Studies in this category are usually not present in-depth validation of the described technical solution but tend to describe a proof of concept by way of example or a sound prototype running. Experience papers, these studies are concerned with reporting the author's experience during the implantation of a new approach. As some selected primary studies did not fit well into the categories above, we extended the classification scheme, by adding a research type category namely Tools. In this category are the studies whose main contribution is the developed tool, often in the form of a research prototype. 

After reading the primary studies, we identified four research methods. As indicated in Table 8, the research methods in the primary studies fall into one of the following categories: (i) Solution proposals, (ii) Evaluation, (iii) Opinion, and (iv) Tools. Solution proposals are by far the most common research method, including 77 studies, 17 primary studies are concerned with some evaluation, and only one study falls into the opinion category. Moreover, the main contribution presented by 11 studies boils down to the implementation of a tool to automate and support group formation, falling into the tools category.

Table 8. Distribution of primary studies by research type.

%---------------------------------------------
% \subsubsection{The answer for RQ3}
%---------------------------------------------
\subsubsection{The answer for RQ3}

RQ2: What type of research is the most common in the field?

A\textsubscript{RQ3}:: The research method used in the majority of the primary papers is solution proposal: 77 studies present group formation approaches that can be applied to CSCL environments. Conversely, the least explored category is opinion. Only one study describes the author’s view about group formation applied to CSCL environments and the implications thereof\todo{verificar este final!}.

%---------------------------------------------
%\subsection{Research Question 4: What are the main venues in which research in this area has been published?}
%---------------------------------------------
\subsection{Research Question 4: What are the main venues in which research in this area has been published?}

As shown in Figure 15, most primary studies are indexed in IEEE (IEEE Xplore) and Elsevier (26 and 24 papers, respectively). ACM Digital Library indexes 19 of the selected papers, 14 papers were retrieved from Scopus, and Springer and Web of Knowledge (WoK) had six papers each. The electronic database with the least amount of selected papers was Engineering Village (5 papers). It is worth mentioning that Springer, WoK, and Engineering Village were the last electronic databases we went through; consequently many of the papers returned by these sources had already been retrieved and selected from other databases. Apart from these studies, 6 selected journal papers were retrieved from SciVerse Hub. 

Figure 15. Frequency of selected primary studies according to the consulted electronic databases.

The selected papers were published in a variety of venues that range from conferences, workshops, to journals. Figure 16 shows an overview of the distribution of primary studies according to their types over the years. By analyzing Figure12, we can see that most efforts in the area were published from 2006 on. Since then, it seems that the topic has gained momentum. As indicated in Figure 16, there was a marked increase in the amount of journal publications in 2011. 

Figure 16. Frequency of selected studies according to the year and the publication of the forum.

As shown in Table 9, 12 venues has contributed with at least two studies that represent 48 publications, while the rest of the studies are spread in 58 different venues (with one publication each). 

Table 9. Main venues in which research about group formation has been published.

%---------------------------------------------
% \subsubsection{The answer for RQ4}
%---------------------------------------------
\subsubsection{The answer for RQ4}

RQ4: What are the main venues in which research in this area has been published?

A\textsubscript{RQ4}: Most primary studies were published in journals. Computers \& Education is the venue with the highest number of selected studies (10), followed by another journal, Computers in Human Behavior and for the proceedings of the Conference on Innovation and Technology in Computer Science Education (ITiCSE), both with six studies. 

%---------------------------------------------
% \section{Discussion}
%---------------------------------------------
\section{Discussion}

Drawing on the information from the primary studies, we identified that group formation usually encompasses three fundamental steps that resemble the input-process-output (IPO) model. In this context, group formation approaches receive information about students as input and then some processing or computation is performed on such information, and the outcome is information on how students should be placed in groups. Basically, these IPO steps can be performed (i) without a computational support (e.g. manually carried out by the instructor), (ii) be partially computerized (e.g. spreadsheets, LMS), or (iii) may be automatically carried out by a computer system.
Moreover, a central topic discussed in some of the primary studies is the strategy used to group students. As mentioned in Section 4, group formation can be carried out in a random fashion (i.e., by chance), or it can be controlled (i.e., criteria-driven group formation). As mentioned, criteria-driven group formation can fall into two categories: spontaneous group formation and controlled group formation. In the former, students are self-assigned to groups, yielding self-selected groups. In the latter, instructors or other agents handle group formation, resulting in instructor-formed groups. Evidence suggests that the main advantages of spontaneous group formation are related to self-motivation, which often leads to a more positive experience throughout the learning activities. However, since self-selected groups are made of participants that share common interests or get along with each other, the same self-selected groups might be repeatedly formed. A clear shortcoming of this approach is that some individuals might be left out. Thus, ostracized students might face difficulties coping with learning activities on their own, leading to unsatisfactory results. 
Clearly, the group formation strategy influences the diversity of groups. Random approaches tend to result in heterogeneous groups. Nevertheless, since groups are formed by chance, it is possible that this approach generates single-gender groups or groups whose most elements belong to a given gender. Another problem with random approaches is that they can also form groups of students that have already mastered the subject in question or groups whose students have struggled to grasp the subject. To enforce group member diversity, therefore making the students interacting with a broad range of peers, students can be clustered in groups based on some criteria. The use of criteria adds a much more fine-grained control over group formation. Moreover, using criteria, it is also possible to create hybrid groups. As mentioned, hybrid group formation approaches employ a multitude of criteria to place students in groups that can be homogeneous and heterogeneous at the same time. However, when more criteria are employed; more information needs to be processed: this means that as the number of criteria increases, the need for computational support also becomes more critical. In our systematic mapping, we found that hybrid group formation is proposed along with tools that automate group formation. 
Devising effective and theoretically sound group formation approaches is a notoriously complex task. Group formation approaches have to take into account many factors as for instance: (i) students’ knowledge, (ii) how students interact with each other, (iii) preferences, needs or limitations of the instructors, environment and students. Given the importance of well-founded group formation approaches, we set out to investigate the rationale behind the approaches proposed in the primary studies. In a way, all primary studies justify the usefulness of their approaches. Some studies, however, fail to describe the theories, concepts, and ideas that underpin their group formation approach. This can be a worrisome finding since, according to the impossibility of justifying either theoretically or pedagogically the selection of participants to compose a group is one of the main weaknesses of the available methods, and a strong reason for teachers’ hesitation in adopting systems with group formation capabilities.
Finally, in order to clarify the classification scheme elaborated; we analyzed the group formation approach proposed by, as follows. In Figure 17, the results of the analysis using such classification scheme are synthesized.

\begin{enumerate}
\item Planning - The paper aims in proposing a systematic criteria-based method for group formation that relies on genetic algorithms to process three characteristics of the students: student knowledge levels, student communicative skills, and an estimate of student leadership skills.
\item Initiative - The system is in charge of clustering the students based on the chosen criteria. 
\item Population diversity – The criteria are used to form both, homogeneous and heterogeneous groups.
\item Distribution approach – A hybrid approach is used since, the authors main goal is to achieve groups that are as similar as possible (homogeneous) to some characteristics of the total sample of students, however also taking into account the heterogeneity of each one. 
\item Computer-supported – Genetic algorithms are used to optimize the odds of using an arbitrary number of student characteristics to form hybrid groups.
\item Rationale – The group formations approach was tested with undergrad students; however, we were not able to identify the justification (neither pedagogic nor instructional) regards the selection of the audience, the learning process, and the learning goals. Still, the study justifies the group formation approach backing up on the benefits of using genetic algorithms to manipulate and manage a considerable amount of students’ characteristics. The paper comments that these three characteristics were based on a literature review.
\end{enumerate}

Figure 17. This figure shows the characterization of the research conducted by Moreno et al. (2012), the characterization follows the classification scheme created in this systematic mapping presented in this section. 

%---------------------------------------------
% \section{Threats to Validity}
%---------------------------------------------
\section{Threats to Validity}

A potential threat to validity is the inability to ensure that all relevant studies were selected. There is a possibility that some studies have been omitted considering that not all electronic databases were consulted. However, all consulted electronic databases are considered important forums of publications, susceptible to contain relevant studies. Thus, it can be considered that the chances that relevant studies have been resolved, or at least been mitigated. Also, we cannot eliminate threats to the validity of the quality point of view of the selected studies, because, during the selection process, we have not assigned scores to the studies.
The research questions and the inclusion and exclusion criteria were drawn up before performing the systematic mapping to ensure an impartial selection process. However, the consistency of the elaborate classification scheme can also pose a threat to validity since the knowledge to elaborate it is often only achieved at the end of the selection process (Dybå and Dingsøyr 2008).
It is important to mention that the difference between the initial quantity of articles and the final selection can be regarded as normal in this type of study. The main objective of a systematic mapping is to provide a broad overview of which have been published in the area of interest. Due this, one important characteristic of kind of study is to avoid imposing many restrictions on the selection of primary studies (Durelli et al. 2013).
Another possible threat to validity results from the operation of some electronic libraries and search systems. In some search engines, despite the use of keywords and wildcards, they often returned items fleeing the scope of the keywords used, therefore demanding further analysis of the undesired studies and manual intervention to eliminate them.

%---------------------------------------------
% \section{Concluding Remarks of the Systematic Mapping}
%---------------------------------------------
\section{Concluding Remarks of the Systematic Mapping}

This paper presented the results of our systematic mapping whose purpose was to provide an insightful overview of the current state of the art on group formation applied to CSCL. The categorization scheme we devised can be used to aid analyzing and understanding proposed group formation approaches. Moreover, it may be useful as a support tool to aid planning new ones. As a future work, we plan to optimize and extend the actual structure of the classification scheme. For example, as Massive Open Online Courses (MOOCs) are proliferating with huge number of participants with different backgrounds, culture and time zones (just to emphasize few points of interest) there is a need for new strategies for group formation in this kind of environment. Traditional approaches used for face-to-face learning or even in traditional e-Learning environments maybe can not be appropriate for MOOCs. Although 25\% of the studies state their approaches are suitable for online learning, only 1\% comment that their approaches are suitable for asynchronous group learning tasks, 7\% for synchronous, 19\% can be used in both scenarios, however, 74\% did not comment if the group interactions describe in the studies were planned to be executed synchronously or asynchronously.
Using the information extracted from the selected studies, we shed light on the most investigated approaches to group formation, the education stages in which these approaches are applied; the most used research methods, and the most common publication types. Besides the growing number of publications and the variety of computational approaches to support group formation, we identified that, the lack of presenting a rationale explaining the group formation approach can be a problem, and furthermore it can be exploited to open significant new opportunities for future research.
Group formation is one of the most important steps during the planning stages of CSCL activities. Due to the many elements that need to be taken into account during group formation, devising a sound group formation approach is considered a complex task. Although research in CSCL has come a long way, most group formation efforts fail to take into account motivational aspects during group formation. Rather, we found that most existing approaches emphasize the use of basic information about the students (e.g., proficiency) during group formation. 
Another contribution of our systematic mapping is that by analyzing its results it is possible to identify the ways in which group formation, when applied in the context of CSCL, has been explored so far. Thus, researchers can refer to this systematic mapping to determine research gaps and decide on future research directions. Finally, we conceive an infographic that summarizes all data presented in this paper, to provide a quick overview of the results presented (Available at: http://infografico.caed-lab.com/mapping/gfc/).









